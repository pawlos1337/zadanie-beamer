\RequirePackage{ifluatex}
\ifluatex
\documentclass[lualatex,aspectratio=54,12pt,]{beamer}
\else
\documentclass[aspectratio=169,15pt,]{beamer}
\fi
%%% aspectratio
% 1610
% 169
% 149
% 141  (1,41:1)
% 54
% 43 (default)
% 32

\usepackage{lipsum}

\input{\jobname .ver}

\usefonttheme{professionalfonts} 

%\usepackage{pgfpages}
%\pgfpagesuselayout{4 on 1}[a4paper,landscape,border shrink=10mm]
%\usepackage{lmodern}

\ifluatex
\usepackage[]{fontspec}
\setsansfont{Carlito}[Numbers=OldStyle]  % lub Numbers=Lining
\else
\usepackage[utf8]{inputenc}
\usepackage[T1]{fontenc}
\usepackage{carlito}
\fi

\title{There Is No Largest Prime Number}
\subtitle{Version: \VCRevisionMod Compiled: \today}
\institute{Department of Mechanics, Materials Science and Engineering}
\date[ISPN ’80]{27th International Symposium of Prime Numbers}
\author[Euclid]{Euclid of Alexandria \texttt{euclid@alexandria.edu}}
%\date{\today}
\pgfdeclareimage[height=\paperheight,width=\paperwidth]{TG}{flower} 
%\titlegraphic{\pgfuseimage{TG}}
\usetheme[vertical,lang=en,hr=true,pagenumbers,navigation]{NewPwr}
%\usetheme{Warsaw}
%\setbeamercolor*{block title}{fg=Logo,bg=Logo!20}
%\setbeamercolor*{block body}{fg=black,bg=Logo!10}

\begin{document}

\begin{frame}
 \titlepage
\end{frame}

%\begin{frame}[plain]
%\makeatletter
%\beamer@pwr@layout
%\makeatother
%\end{frame}

\frame{\tableofcontents}

\section{Raz}

\begin{frame}
 \frametitle{There Is No Largest Prime Number}
 \framesubtitle{The proof uses \textit{reductio ad absurdum}.}
 \begin{theorem}
  There is no largest prime number. 
  \end{theorem}
  \begin{proof}%[Dowód]
 \begin{enumerate}
  \item<1-| alert@1> Suppose $p$ were the largest prime number.
  \item<2-> Let $q$ be the product of the first $p$ numbers.
  \item<3-> Then $q+1$ is not divisible by any of them.
  \item<1-> But $q + 1$ is greater than $1$, thus divisible by some prime
        number not in the first $p$ numbers.
 \end{enumerate}
 \end{proof}
\end{frame}

\begin{frame}{A longer title}
 \begin{itemize}
  \item one
  \item two
 \end{itemize}
\end{frame}

\section{dwa}

\begin{frame}{Lista punktowana}
 \begin{itemize}
  \item
        Item 1
  \item
        Item 2
  \item
        Item 3
  \item
        \alert{Alerted item}
 \end{itemize}
\end{frame}

\subsection{trzy}

\begin{frame}
 \frametitle{Lista numerowana}
 \begin{enumerate}
  \item
        Item 1
  \item
        Item 2
  \item
        Item 3
  \item
        \alert{Alerted item}
 \end{enumerate}
\end{frame}

\subsection{cztery}

\begin{frame}
 \frametitle{Blok}
 alertblock:
 \begin{alertblock}{Tytuł}
  Zawartość
 \end{alertblock}

 exampleblock:
 \begin{exampleblock}{Tytuł}
  Przykład Przykład Przykład Przykład Przykład Przykład Przykład Przykład Przykład Przykład Przykład Przykład Przykład Przykład Przykład Przykład Przykład Przykład Przykład Przykład
 \end{exampleblock}

 zwykły blok
 \begin{block}{Tytuł}
 Przykład, przykład, przykład…
 \end{block}

 alertenv:
 \begin{alertenv}<2>
  (environment contents)
 \end{alertenv}
\end{frame}

\part{Lipsum}

\begin{frame}
\partpage
\end{frame}

\begin{frame}[allowframebreaks=0.97]{Długi Tekst}
 \lipsum
\end{frame}

\begin{frame}[plain,c]
plain slide
\end{frame}

\begin{frame}{Ostatni slajd}
ala ma 
\end{frame}
\end{document}
